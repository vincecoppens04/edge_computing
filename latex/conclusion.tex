
\section{Benefits of edge--cloud computing}
Running computation close to the source remains the core advantage of edge systems, reducing latency and avoiding unnecessary transport of high-volume data. The shift from isolated edge nodes to edge--cloud architectures combines this locality with the manageability of cloud platforms. In practice, containerisation and lightweight Kubernetes distributions make large fleets more uniform and easier to update, an essential requirement as deployments scale. Local filtering also improves privacy and reduces bandwidth cost. These elements were all reflected in the prototype: the Raspberry~Pi generated continuous data, while the Mac-based edge cloud handled inference and selective upstream publication.

\section{Challenges and limitations}
The same decentralisation that increases resilience introduces operational and security complexity. Heterogeneous hardware, fragmented standards, and the lack of mature interoperability across frameworks remain major obstacles. Our experiment illustrates this in a modest way: even basic configuration differences, OS incompatibilities, and broker-level issues already complicate deployment. At industry scale, the problem is orders of magnitude greater. Although Kubernetes and KubeEdge promise a unified control plane, their integration across constrained devices is still uneven, and vendor ecosystems diverge rather than converge. Security remains a structural weakness: dispersed devices are physically exposed, subject to uneven patching, and difficult to monitor.

\section{Conclusion}
Edge computing has evolved into an edge--cloud model because centralised architectures cannot keep pace with the growth of distributed AI workloads. The combination of local inference and cloud coordination offers clear benefits, but the ecosystem is still maturing. Fragmented standards, operational overhead, and heterogeneous hardware limit frictionless deployment. Our implementation shows that, although a functional pipeline is achievable with open-source tools, achieving true scalability, interoperability, and secure lifecycle management remains an open challenge. Edge cloud is therefore not a replacement for the cloud, but an architectural compromise driven by practical constraints and still dependent on further standardisation and refinement.